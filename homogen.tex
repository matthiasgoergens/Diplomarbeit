% Plan: Homomorphismus von Hanno einbauen,
%       Beispiel und Allgemeines in Einleitung trennen.
%       Historie korrigieren (Umlauftage richtig darstellen!)

\documentclass[a4paper]{amsart}
\usepackage{a4}
\usepackage{charter}

\usepackage[english]{babel}
\usepackage[utf8]{inputenc}
\usepackage{amssymb}
\usepackage{amsfonts}
\usepackage{units}
\usepackage{amsmath}
%\usepackage{pdfsync} % does not seem to work currently.

\newtheorem{theorem}{Theorem}

\theoremstyle{definition}
\newtheorem{defn}[theorem]{Definition}


\theoremstyle{remark}
\newtheorem{remark}[theorem]{Remark}

\author{Matthias Görgens}

% ---- LATIN ----
\def\etal{\emph{et~al.}}
\def\ie{\emph{i.e.}}
\def\eg{\emph{e.g.}}
\def\vitae{vit\ae{}}
\def\apriori{\emph{a~priori}}
\def\aposteriori{\emph{a~posteriori}}

% ---- FRENCH ----
\def\naive{na\"{\i}ve}
\def\Naive{Na\"{\i}ve}
\def\naively{na\"{\i}vely}	% Okay, I know, this isn't French.
\def\Naively{Na\"{\i}vely}


% Other
\def\Cpp{C\raisebox{0.5ex}{\tiny\bf++}}

\newcommand\mpar[1]{\marginpar {\flushleft\sffamily\small #1}}
\setlength{\marginparwidth}{3cm}
\setlength{\marginparpush}{1cm}

\newcommand{\ol}[1]{\overline{#1}}
\newcommand{\todo}[1]{\mpar{#1}}
%\newcommand{\todo}[1]{\footnote{#1}}

\DeclareMathOperator{\In}{in}
\DeclareMathOperator{\Out}{out}

%für ein/aus-gehende kanten
\newcommand{\ina}{\ensuremath{\vec{\In}}}
\newcommand{\outa}{\ensuremath{\vec{\Out}}}

%für knoten ein/aus-gehender kanten.
\newcommand{\inv}{\ensuremath{\dot{\In}}}
\newcommand{\outv}{\ensuremath{\dot{\Out}}}

\newcommand{\reals}{\ensuremath{\mathbb{R}}}
\newcommand{\naturals}{\ensuremath{\mathbb{N}}}
\newcommand{\rationals}{\ensuremath{\mathbb{Q}}}

\newcommand{\Hom}{\ensuremath{\mathrm{Hom}}}
\newcommand{\Sym}{\ensuremath{\mathfrak{S}}}

\newcommand{\abs}[1]{\ensuremath{\left|#1\right|}}
% lr := left right
\newcommand{\lr}[1]{\ensuremath{\left( #1 \right)}}
% lrX := left right fuer spitze Klammern
\newcommand{\lrX}[1]{\ensuremath{\left< #1 \right>}}
% lrM := left right fuer Mengen
\newcommand{\lrM}[1]{\ensuremath{\left\{ #1 \right\}}}


\usepackage{hyperref}
\begin{document}

\date{\today}
\title{Homogeneous vehicle schedules}

\begin{abstract}
  Although most railroad timetables remain in use for half a year,
  they tend to repeat themself each week.  Thus one can also repeat
  their implementations in rolling stock, the vehicle schedulings,
  each week.

  But timetables show even more structure: All the working days of the
  week are usually identical.  This work sets out to investigate how
  to preserve this similarity in vehicle schedulings.  We define
  homogeneous vehicle schedulings, establish a measure of partial
  homogeneity and search for ways to efficiently find the most
  homogeneous vehicle scheduling for a given timetable.
\end{abstract}

\maketitle
\tableofcontents
\todo{unterschied zwischen fahrplan und umlaufplan im abstract ausarbeiten.}



\section{Introduction}
Unlike individual traffic in cars, railroads in Germany are run by
only a few companies and of those Deutsche Bahn (and its subsidaries)
control a huge majority.  In principle the planning required can be
centralized for coordination.  Thus one can exploit effects of
synergy.  Even relatively modest relative improvements can have a huge
absolute impact.  So it is worthwhile to solve the encountered
problems to optimality.

One large block---or even the largest---of opportunities for
mathematical planning in this contexts lies in the area of scheduling.
Currently at Deutsche Bahn the system for planning schedules is
multi-tiered.  This paper has been written while working at department
GSU1 of DB Mobility Logistics.  The focus at GSU1 is on
improving the process of (strategic) vehicle schedulings.

As a customer one only sees the timetables, that specify when and
where a train will arrive and depart to take on or load off cargo or
passengers.  One does not care too much which specific physicial train
one is using or what it does before and after the trip.

Naturally the railroad company - in contrast - does care about what
happens to a train after it finishes one trip.  Normally it goes on to
provide another trip to paying customers.  From time to time there
will also be maintenance stops.

We will only be concerned with strategic planning, that is about the
connection between trips on a regular schedule planned in advance.  In
the field the best laid plans often need patching on the fly, because
of unforeseen circumstances like accidents.  An activity called
operational planning and outside the scope of this work.  Also we will
view the timetables of trips offered to the customers as fixed to
reflect the conditions encountered at GSU1.

\todo{Ein abregenzendes Wort zu timetable und operationeller Planung
  im Vergleich.}

Introducing our first bit of notation we call trips offered to the
customer \textit{service trips} (or Nutzfahrten in German) and other
trips will be called \textit{deadhead trips} (Leerfahrten in German).
Deadhead trips may be necessary because a station may have an unequal
number of service trips departing and arriving over the course of a
week.  Or the number is equal, and the station could keep the balance
in theory, but only with long unproductive standing times for the
trains.

The objective of strategic planning is to reduce the cost of
implementing a fixed timetable.  The number of locomotives needed
contributes the biggest chunk to those costs.  Avoiding deadhead trips
also saves some money, as they cost man-hours, fuel, and bring the
next obligatory maintenance closer.  Because most types of maintenance
aren't scheduled in fixed time intervals, but after certain distances
travelled.

Most timetables are in use for half a year.  They tend to repeat every
week.  At the current state of the art, strategic vehicle schedule
planning ignores the start and end of the period and pretends to work
on an infinite cycle of identical weeks.  This rolls up a long
schedule to a more tractable cycle of just one week.

Strategic vehicle schedule planning in essence takes in cyclic
timetables (plus possible non-cyclic extra trips) and puts out vehicle
schedules to the operational planners, who \eg{} manage the transition
between schedules.

Deutsche Bahn has only recently brought sophisticated math to bear on
these issues.  There are legacy heuristics in place that solved (and
do solve) scheduling problems, but with no guarantees of optimality.
Also customs and preferences have developed.

For example planners prefer their schedule to be as homogenous as
possible.  Because this is what they got by default with the old
methods, and thus a certain sense of simplicity and aesthetics has
developed.  Typically, in freight logistics for example, a huge part
of all trips repeats with the same time and route every Tuesday,
Wednesday and Thursday.  So these three days get scheduled as one
cycling day, and this is where the homogeneity comes from.  To produce
a working schedule, this single day cycle is then unrolled to a
three-day cycle, and the planners add the other trips occurring in the
middle of the week.  Finally the cycle is cut, and the other four days
of the week and their trips are inserted.

To make the schedule usable, one has to answer further questions, like
how to deal with trips that do not repeat every week.  And how to
change from one schedule to another.  But these are beyond the scope
of this stage of planning at Deutsche Bahn, and beyond the scope of
this work.

At each step the above, where new trips are inserted into the
proto-schedule, the old trips can get re-scheduled, if this makes the
schedule cheaper to implement.

The new methods based on mixed integer programming consider the whole
week right from the start.  Apart from the partly repeating timetable
they have no reason to use similar vehicle schedulings on the working
days.  But planners have come to price the homogeneity as an end in
itself, because their preferred visualization of schedules looks more
pleasing in the presence of homogeneity.  \todo{Stimmt der Grund?}
\todo{Add more about blocks. Visualization.}

\todo{Noch mehr Grundlagen.  Ausfuehrungen zu Betriebsablauf bei der
  Bahn.}

\section{Thanks}
Thanks to Hanno Schülldorf, Volker Kaibel and Andreas Huck for their
help.  \todo{Noch ein paar mehr Leute aufführen.}

\section{Some definitions}

\todo{Noch mehr mathematische Grundlagen?  Column and Row generation? Ja.}

\todo{Hanno schlägt vor, Beispiel (Woche = 7 Tage) und allgemeine
  Theorie zu trennen.

  Plan: Nutze Woche als technischen Term (so wie Byte), und lege erst
  später fest, dass \(w := 7\).}

\subsection{Timetables}

Let \(V\) be the set of all stations \eg{} ordinary railway stations,
maintenances shops.  The rail network of connected stations will be
modelled as a digraph \(G = (V,A)\), where \(A \subseteq V \times V\).
A \label{trip} \textit{trip} starting in one station \(v \in V\) and ending in another
station \(w \in V\) can be characterized by a tuple \((v, t_v, w, t_w)
\in \left(V \times \rationals\right)^2 \),
where \(t_v\) and \(t_w\) are the departure and arrival times.
\newcommand\dep[0]{\ensuremath{\operatorname{dep}}}
\newcommand\arr[0]{\ensuremath{\operatorname{arr}}}
\begin{defn}[Departure and Arrivals]
  \begin{align*}
    \dep, \arr \colon & \lr{V \times \rationals}^2 \to \lr{V \times \rationals} \\
    & x = \lr{\dep \lr{x}, \arr \lr{x}}
  \end{align*}
\end{defn}
We call a set of trips \(T \subseteq \left(V \times
  \rationals\right)^2\) a timetable.  (In practise \(T\) may also be a
multiset and will carry more information.)

We want to focus our attention on cyclic timetables.
\begin{defn} A timetable \(T\) is \label{cyclic} \textit{cyclic}, iff
  there exists a period \(w \in \rationals^+\), so that for each trip
  \((v, t_v, w, t_w) \in T\) the trips \(\{(v, t_v+n w, w, t_w +n w)
  \mid n \in \mathbb{Z}\}\) are also in \(T\).
\end{defn}
We will assume that no trip takes longer than one cycle and
we will identify cyclic timetables with their image under the canonic
morphism
\[\left(V \times \rationals\right)^2 \to
\left(V \times \rationals/{\left(w \mathbb{Z}\right)}\right)^2\text{.}\]

\subsection{The matching model for schedules}

A timetable only fixes the service trips offered by the rail operator.  But
it does not specify where the locomotives and cars in use come from or
where they go afterwards.  Naturally one wants to re-use rolling stock
from one trip for another one.  So one seeks to match arrivals from
one trip with following departures from another trip.

\begin{defn}[Matching formulation]
  Given a timetable \(T \subseteq \left(V \times
    \rationals\right)^2\), a perfect matching \(M \subseteq \arr \lr{T} \times \dep \lr{T}\)
  describes a schedule to implement \(T\) using the matching formulation.
\end{defn}
\begin{defn}[Chains-of-trip formulation]
  Given a timetable \(T \subseteq \left(V \times \rationals\right)^2\)
  and a schedule in matching formulation \(M \subseteq \arr \lr{T}
  \times \dep \lr{T}\) that implements it,
  define the chains-of-trip formulation \(C \in \Sym\lr{T}\)\footnote{where \(\Sym \lr{T}\) is the symmetric group over \(T\)} of that schedule as follows:
  \[C \lr{x} = y\text{, iff }\lr{\arr\lr{x},\dep\lr{y}} \in M\text{.}\]
  This can also be viewed as a \(\lrM{0,1}\)-network-circulation.
\end{defn}

The two formulations above are equivalent if all times are distinct.
If necessary extending the definitions to include unique labels for
all trips (and their departures and arrivals) preserves this property
for all cases.  But we want to keep the presentation simple.

\subsection{Costs}

Not all schedules are equal.  In general different schedules implement
a given timetable at different costs.  One component of the costs is
proportional to the time each locomotive is in us (\eg{} interest on
capital, deprecation charges).  The other component is caused
primarily by the distance travelled by each locomotive, mainly because
this causes wear and maintenance stops are scheduled after fixed
intervals of distances travelled.  Man-hours for the locomotive
drivers \todo{Lokführer auf Englisch?} are a special case, because
they only occur while travelling a distance, but are proportional to
the time this takes.

Of course the service trips in our timetable also cause costs.  But
they are the same for every schedule, so we ignore them, since we are
only interested in differences between schedules.  The schedule can
include other so called deadhead trips, if it matches an arrival at
and departure from different stations.

All in all we can satisfy those requirements by assigning a specific
cost to every edge of the matching graph, and sum the costs of the
edges used:
\begin{defn}[Costs in the matching model]
  Given a cost vector \(c \in \rationals^{\arr \lr{T} \times \dep\lr{T}}\) the cost \(C\) of a schedule \(M\) is:
  \[C \lr{M} := \sum_{x \in M} c \lr{x}\]
\end{defn}

\begin{remark}
  There is an alternative, but equal, formulation in use at Deutsche
  Bahn that allocates the components of the costs that are
  proportional to the time the locomotives are in use, not to each
  match or trip, but only to the matches and trips that span a
  specific point in time of the cycle (\eg{} Sunday, 24.00), the so
  called cycle jump\todo{Ist period jump ein vernünftiger Begriff?}.
  This works because we can view the scheduling as a network
  circulation problem, in which flow is preserved.  This allocation
  concentrates costs on matches spanning the period jump and on the
  deadhead trips, leaving most matches without associated costs.
  Often this makes \textsc{Cplex} solve the problem faster.
\end{remark}

\subsection{The block formulation gives rise to homogeneity}
\todo{Blockdarstellung auf englisch?}

At Deutsche Bahn another approach is also in use.

\subsection{Homogeneity in the matching model}

In practise timetables repeat every week \ie{} \(w = 7 \textrm{ days}\).
This work will deal with timetables that show more structure: Fix an
interval \(d := \frac{w}{n}\) with \(n \in \naturals^+\) (\eg{} for
days in a week, let \(n=7\)).  We partition the trains into
equivalence classes that share the same arrival and departure
stations, and whose arrival and departure times only differ by an
integral multiply of \(w\) (\ie{} whole days):
\begin{equation}
  \left[\left( v, t_v, w, t_w \right) \right]_d := \{(v, t_v+n d, w, t_w +n d) \mid n \in \mathbb{Z}\}
\end{equation}
Let \(C\) be the set of all those classes.

Ideally each class has exactly one member for each day of the week.
That would allow \(w = 1 \textrm{ day}\) instead of \(w = 7 \textrm{ days}\)
and enable a schedule that repeats perfectly every day.  But more
typical we observe a lot of classes in a timetable with one member
each day from Monday to Friday and none on the weekends.  Or even less
regular arrangements.

We will define a distance function \(\Delta_S (s,t)\) on paths in
\(S\) as the sum of arc lengths \(d\) in the path between the
trains \(s\) and \(t\) or \(\infty\) if no such path exists.  We can
talk about \emph{the} path \(P \subseteq S\) between \(s\) and \(t\),
because \(G_S\) consists only of disjoint cycles.
\[\Delta_{S}\colon T \times T \to \rationals^+_0 \cup \{\infty\}\]
\begin{equation}
\label{defDelta}
\Delta_S (s,t) = \begin{cases}
\sum_{(v,w)\in P} d(v,w) & \textrm{if there is a path \( P \in S\) between \(s\) and \(t\)}\\
\infty & \textrm{else}
\end{cases}
\end{equation}

\todo{Bezeichnungen!  Pfeile über t?}

\todo{Mehrwöchige Fahrten:  Zurückgehen auf die ursprüngliche nicht-zyklische Version!}

Of course we have only moved the problem.  We still have to define the length of a single arc:  \(d(t, \ol{t})\) shall be the minimal time that a vehicle needs to provide \(t\) and be ready for the departure of \(\ol{t}\):
\[
d ((\cdot, t_{\textrm{dep}}, \cdot, t_{\textrm{arr}}), (\cdot, \ol{t_{\textrm{dep}}},\cdot, \cdot)  :=
\min (\{\ol{t_{\textrm{dep}}} - t_{\textrm{dep}} + n w \mid \ol{t_{\textrm{dep}}} - t_{\textrm{arr}} + nw \geq 0, n \in \mathbb{Z}\} \cap \rationals^+_0)
\]

In most cases this will be just the difference \(\ol{t_{\textrm{dep}}}
- t_{\textrm{dep}}\), but we may need to add a whole number of weeks
if \(\ol{t}\) departs before \{t\} arrives.  Also a more general approach
has to be taken when trains take more than one week from arrival to
departure.

To guarantee homogeneity one might \naively{} propose that all members
of one class \(c_1\) only be succeeded by members of the same class
\(c_2\).  But that would not be practial, because \eg{} a class that is
offered all week would be banned from being connected to a class of trains that
are not offered on the weekend.

\todo{motivation.  fällt aus der luft.. zitat der begriffsbestimmung.}
So we introduce a weaker condition.  A schedule \(S\) is called
homogenous iff:
\begin{defn} Given
  \begin{itemize}
  \item classes \(z, \ol{z} \in C\)
  \item and trains \(r_1, r_2 \in z\) and \(\ol{r}_1,\ol{r}_2 \in \ol{z}\)
  \item with \( d (r_1, r_2) = d(\ol{r}_1, \ol{r}_2) \),
\end{itemize}
  then either there is no path between neither \(r_1\) and
    \(\ol{r}_2\) nor \(r_2\) and \(\ol{r}_2\) in \(S\); or they are connected by paths of the same length:
    \begin{equation}
      \label{homoEq}
      \Delta_S (r_1, \ol{r_1}) = \Delta_S (r_2, \ol{r_2})\textrm{.}
    \end{equation}
\end{defn}


\todo{motivation fuer definition}
\begin{defn}
For arcs in a digraph \(G=(V,A)\) we will use the following notation:
\begin{align*}
\ina\colon  V &\to \mathcal{P}(A) \\
v &\mapsto \left(V \times \{v\}\right) \cap A\\
%\end{align*}
%\begin{align*}
\outa\colon  V &\to \mathcal{P}(A) \\
v &\mapsto \left(\{v\} \times V\right) \cap A\\
%\end{align*}
%\begin{align*}
\inv\colon  V &\to \mathcal{P}(V) \\
v &\mapsto \left\{ u \colon \left(u,v\right) \in \ina(v) \right\}\\
%\end{align*}
%\begin{align*}
\outv\colon  V &\to \mathcal{P}(V) \\
v &\mapsto \left\{w \colon \left(v,w\right) \in \outa\left(v\right) \right\}\\
\end{align*}
\end{defn}

\section{Tori}

\todo{Hannos Ideen: Morphismen zwischen Torus (Umlauftag, Verkehrstag)
  und Nutzfahrketten aufschreiben.  Homogene Umläufe entsprechen
  \((+1,+1)^n\) trips auf dem Torus, die aber nicht notwendig
  partitionieren.  (Es kann Patz auf dem Torus bleiben.)

  I.A. Mehrere Tori nötig.  Wenn nur komplett gefüllte Tori, dann
  haben wir echte Homogenität.
  
  Homogenität kann (immer) hergestellt werden durch Teilen von
  Zügen.  (Minimiere diese Zahl, wenn nötig.)  Einfacher zu
  handhaben als abweichende Wenden.}

\todo{Notation vernünftig einführen.}

The traditional view, that naturally produces homogeneous vehicle
schedulings, puts them into a kind of torus; \ie{} a cartesian
product of two cyclic groups.  One group consists of the days of the
week (or whichever period it is, that is repeated).  The other group
is more abstract and is basically used to give different trips on the
same weekday different coordinates.  The first group is called
Verkehrstage in German.  We will employ the simpler term weekdays or
days of the week.  The second group is called Umlauftage, and we will
call them virtual days.

Pretending for the moment that service trips do not span more than one
day, each trip gets assigned its day of the week and a virtual day as
its coordinates on a torus.  A locomotive will start implementing the
schedule on a certain coordinate, too.  It will do all trips with this
coordinates.  Thus trips that can't be done by one locomotive can't
have the same coordinates.  They need to differ in their virtual day.
After the day has passed, the locomotive will go on to the trips with
coordinates \(\lr{+1,+1}\) from where it was.

\todo{Is \(\operatorname{lcm}\) (least common multiple) the
  common term for kgV?}

Just like each trip gets repeated every week, the schedule of a
single locomotive repeats every \(\operatorname{lcm} \lr{w, u}\) days,
where \(w\) is the number of days in a week and \(u\) is the order of
the group of virtual days for this locomotive.

In general a torus needs as many locomotives as it has virtual days.
Only tori that consist of a single cycle, that is \(\gcd{} \lr{w,u} =
1\) or \(\lrX{\lr{+1,+1}} = W \times U\), are considered proper in the
existing literature at Deutsche Bahn.  But we will extend the notion
to tori partitioned into more than one orbit under \(\lrX{\lr{+1,+1}}\).

A schedule is said to be homogeneous if for any given train all its
trips happen on the same virtual day.  If homogeneity causes more
locomotives to be used, the planners violate it in two different ways:
Trains are split up into different virtual days, and locomotives may
continue to a coordinate different from adding \(\lr{+1,+1}\).


\section{An infinite integer linear program}
\label{infIP}

Let \(T\) be a timetable.  And let \(G_\mathbb{S} := (T, \mathbb{S} =
T \times T) \) be its associated complete graph.  A schedule \(G_S =
(T, S \subseteq \mathbb{S})\) will be a subgraph of \(G_\mathbb{S}\).


For the integer program we introduce the variable \(m\) with the
following domain and interpretation:
\begin{align}
  m & \colon T \times T \to \{0,1\} \\
  m & \cong \mathbf{1}_{S}
\end{align}
Where \(\cong\) means ``defined as'' or ``to be interpreted as''.

The following two equations capture the requirement that \(G_S\)
consists of disjoint cycles.  For all \(r \in T\):
\begin{align}
  \label{matchBedingung}
  \sum_{a \in \ina(r)}  m(a) = \sum_{a \in \outa(r)} m(a) = 1
\end{align}

To reflect the values of \(\Delta_S\) in the integer program we use the binary variables \(\delta\):
\begin{align}
  \delta \colon T \times T \times \rationals & \to \{0,1\} \\
  \delta\left(r, q, \Delta_S\left(r,q\right)\right) &\cong 1 \\
  \label{onlyOne}
  \sum_{z \in \rationals} \delta(r, q, z) &\leq 1
\end{align}

For every simple path \(P \subseteq \mathbb{S}\) that starts in \(r\) and ends in \(q\) we have:
\begin{align}
\label{zwingHoch}
\sum_{a \in P} (m(a) - 1) \leq \delta \left(r,q, \sum_{a \in P} d \left(a\right)\right) - 1 \textrm{;}
\end{align}
\ie{} if \(P \subseteq S\) then length of \(P\) gives
the distance between its endpoints in \(S\).

And for every \(r\)-\(q\)-cut \(C \subseteq T\) (with \(r \in C\)):
\begin{align}
\label{zwingRunter}
  \sum_{a \in C \times (T \setminus C)} m(a) \geq \sum_{z \in \rationals} \delta (r,q, z)
\end{align}

% \(G_S=(T,S\subseteq T^2)\)


% Aufbau: Modell oder Anwendung?
% Anwendung: Umlaufplanung etc, Abschätzung anderer Möglichkeiten.  Anwendung zur Motivation.
% Welcher Umfang zur Anwendung?  Literatur.  Allgemeine Beschreibung?  Irgendeine Referenz wird es schon geben.

% relation durch Übersetzung von Nutzfahrt ersetzen.

\todo{welche constraints?  labels und refs.  convex set vs gitterpunkte?}
Those constraints define a convex set.  Alas we can't use them
directly, since there are an infinite number of constraints.  But
fortunately they are constructed for use with row generation.  We will
show that there is a finite subset of constraints that define the same
polytope.  And we will show how to define a polynomial oracle to find
violated constraints.

Now we only have defined a way to measure time-distances in the
vehicle schedule in a linear integer program.  We still have to
incorporate the actual homogeneity constraints laid out in equation
\ref{homoEq}.  But this is simple as we only have to convert
equation \ref{homoEq} from using \(\ol{\Delta}_S\) to an expression in terms of \(\delta\).

\begin{defn}
Given
\begin{itemize}
\item classes \(z, \ol{z} \in C\)
\item and trains \(r_1, r_2 \in z\) and \(\ol{r}_1,\ol{r}_2 \in \ol{z}\)
\item with \( d (r_1, r_2) = d(\ol{r}_1, \ol{r}_2) \),
\item then for every \(z \in \rationals\)
\begin{equation}
\label{homoEqLin}
  \delta    (r_1, \ol{r_1}, z) = \delta (r_2, \ol{r_2}, z)\textrm{.}
\end{equation}
\end{itemize}
\end{defn}
\section{Finitizing the integer linear program}
\label{finitizing}
\todo{Motivation!}

The linear integer program of section \ref{infIP} describes the
problem of finding a homogeneous vehicle schedule.  It uses an
infinite number of variables.  That is hardly tractable.  But we can
reduce the number of variables from infinite to finite.

Equation \ref{matchBedingung} holds for all \(r \in T\) and \(T\) is
already finite.  We declared \(\Delta_S\) to map into \(\rationals^+_0 \cup
\{\infty\}\), but we can restrict the image even further:
\begin{equation}
\label{einfacher}
\Delta_S (s,t) \in \{d (s,t)\} \oplus w \naturals_0
\end{equation}
where \(\oplus\) is the Minkowski sum.
\todo{Proof why \ref{einfacher} (=equation above) holds!}

Thus all \(\delta (s, t, z)\) with \(z \notin \{d (s,t)\} \oplus w
\mathbb{Z}\) vanish.  Furthermore because of the way \(\Delta_S\) is
defined in equation \ref{defDelta} we can bound it from above and below. Call
\begin{equation}
M := \sum_{(r,q) \in T \times T} d (r,q)
\end{equation}
and note
\[0 \leq \Delta_S(s,t) \leq M \textrm{.}\]


Now for every \(s,t \in T\) the relation in an infinite number of
variables described by constraint \ref{onlyOne} can be replaced by the finite:
\begin{align}
  \label{onlyOneFinite}
  \mathop{\sum_{z \in \left\{d (s,t)\right\} \oplus w \naturals_0}}_{z\leq M} \delta(r, q, z) & \leq 1 
\end{align}

A similar transformation turns constraint \ref{zwingRunter} into
\begin{align}
  \label{zwingRunterFinite}
  \sum_{a \in C \times (T \setminus C)} m(a) & \geq \mathop{\sum_{z \in \left\{d (s,t)\right\} \oplus w \naturals_0}}_{z\leq M} \delta (r,q, z) \textrm{.}
\end{align}

To simplify our notation we will introduce
\begin{equation}
\label{defL}
L := \{0, 1, \dots,\left\lceil \frac{M}{w} \right\rceil \}
\end{equation}
and substitue \(D\) for \(\delta\):
\begin{align}
  D \colon & T \times T \times L  \to \{0,1\} \\
          & (s,       t,        n)                              \mapsto \delta (s,t, d(s,t) + n w)
%  D (s,t,n) := 
\end{align}
Now we can rewrite \ref{onlyOneFinite} and \ref{zwingRunterFinite}:
\begin{align}
  \label{onlyOneFiniteN}
  \sum_{n \in L} D (r, q, n) & \leq 1 \\
  \label{zwingRunterFiniteN}
  \sum_{a \in C \times (T \setminus C)} m(a) & \geq \sum_{n \in L} D (r,q, n)
\end{align}

Putting \ref{zwingHoch} into the new form poses bigger
problem. \todo{Warum?  Hier erklärung einfuegen.}  For every path \(P
\subseteq \mathbb{S}\) that starts in \(r\) and ends in \(q\) we have:
\[
%\sum_{a \in P} (m(a) - 1) \leq \delta \left(r,q, \sum_{a \in P} d \left(a\right)\right) - 1  \\
%\sum_{a \in P} (m(a) - 1) \leq D \left(r,q, \left[ \sum_{a \in P} d \left(a\right) \right] \right) - 1 \\
\sum_{a \in P} (m(a) - 1) \leq D \left(r,q, \frac{\left(\sum_{a \in P} d \left(a\right)\right) -d(r,q)}{w}\right) - 1  \\
\]
and thus:
\begin{align}
\label{zwingHochFiniteN}
\sum_{a \in P} (m(a) - 1) \leq D \left(r,q, \sum_{a \in P} \frac{ d \left(a\right)}{w} - \frac{d(r,q)}{w}\right) - 1
\end{align}

\todo{Vorher könnten wir n fuer jeden Bogen unabhäging wählen.  Jetzt
  muessen wir eine Abhängigkeit fordern.  Gluecklicherweise können wir das statisch
  bestimmen!

  Was ist die Länge von \(P\) in terms of n in abhängigkeit der
  zwischenschritte?  Und das hat auch geklappt in \ref{zwingHochFiniteN}}

\todo{Formuliere das verendlichte Problem konkret-komplett.  Und fuege die eigentliche homogenitätsbedingung hinzu!}

We can express the constraint \ref{homoEqLin} in terms of \(D\), too:
\begin{itemize}
\item Given classes \(z, \ol{z} \in C\)
\item and trains \(r_1, r_2 \in z\) and \(\ol{r}_1,\ol{r}_2 \in \ol{z}\)
\item with \( d (r_1, r_2) = d(\ol{r}_1, \ol{r}_2) \),
\item then for every \(n \in L\)
\begin{equation}
\label{homoEqFin}
  D    (r_1, \ol{r_1}, n) = D (r_2, \ol{r_2}, n)\textrm{.}
\end{equation}
\end{itemize}

\section{Estimating the size of the finite integer linear program}
In section \ref{infIP} we have introduced an integer linear program to
express the problem of homogeneity in vehicle schedulings.  Alas that
program was infinite and thus intractable.  In section
\ref{finitizing} we have finitized the problem.  In theory one could
now feed it into a generic linear program solver and be done with it.

But even the finitized problem suffers from an exuberant size.

How bad is the situation?  The system is made up of binary variables.
There is one variable \(m\) for each arc in the complete graph
\(G_\mathbb{S}\).  So this part takes \(O\left(\left|
    T\right|^2\right)\) variables, where the \(T\) was the number of
service trips.  To model the (possible infinite) length of the path in
\(G_S\) between any two service trips, \(D\) is used.  \(D\) spans
\(O\lr{\abs{T}^2 \abs{L}} =
O\lr{\abs{T}^2 \frac{M}{w}} \) binary variables.  We
take \(w\) to be a constant (usually seven days) and estimate \(M\):
\begin{equation}
M = \sum_{(r,q) \in T \times T} d (r,q) \leq \left|T\right|^2 \max_{(r,q) \in T \times T} d(r,q)
\end{equation}
In practise in Central-Europe trips--- both deadhead and service trips
---are much shorter than half a week.  So we can further estimate:
\begin{equation}
M \leq \left|T\right|^2 w
\end{equation}
Thus \(O\lr{\abs{T}^4}\) variables constitute \(m\), which
dwarves the contribution of \(D\).

\todo{Referenzen zu den relevanten Gleichungen und Namen einfuehren
  fuer die O() schätzungen.}

All in all the problem lives in relatively small polynomial number of
binary variables.  However the number of constraints is exponential:

Equations \ref{matchBedingung} describe the conservation of flow.
They take up a modest \(O\lr{\abs{T}}\) rows.  Equations
\ref{onlyOneFiniteN} restrict \(D\) to at most one \(1\)-entry per
pair of service trips, thus ensuring an isomorphism between \(D\) and
\(\ol{\Delta}_S\).  This takes \(O \lr{\abs{T}^2}\) rows.  So far the
number of constraints has been constrained polynomial in \(\abs{T}\).

Equations \ref{zwingRunterFiniteN} come in one version for each cut
\(C \subseteq T\).  Since there are \(O \lr{2^{\abs{T}}}\) cuts, we
need as many rows.  The number of simple paths \(P \subseteq
\mathbb{S}\) that connect every pair of service trips \(r\) and \(q\)
determines the number of rows needed for implementation of constraint
\ref{zwingHochFiniteN}.  It is of equally staggering proportions:
\(O\lr{2^{\abs{T}} \abs{T}!}\) in a lose estimate from above, since
there are \(O\lr{2^{\abs{T}}}\) subsets of service trips that can be
ordered in less than \(O \lr{\abs{T}!}\) ways each.

Taking everything together, we need \(O\lr{\abs{T}^4}\) variables and
\(O\lr{2^{\abs{T}} \abs{T}!}\) constraints.  The next section will
deal with ways to reduce the number of constraints (and incidentally
the number variables, too).

\section{Delayed row generation}
We can start solving with only the equations \ref{matchBedingung} that
guarantee a flow free of sources and sinks and the variables \(m\).
All other variables are treated as parameters set to zero.

We analyse the solutions obtained.  The variables \(m\) specify a
graph made of cycles.  For every cycle we can get the distance
\(\Delta\) between all of its nodes by straight-forward counting in
time \(O(T^2)\) \todo{Details zum Zählen.} (or employing Bellman-Ford
etc).  All distances between nodes in different cycles are \(\infty\)
by definition \ref{defDelta}.  \todo{Perhaps use lazy evaluation to enumerate
  all flaws.  Pick the first.}

First check \ref{onlyOne} for every pair of vertices.  If a pair, say
\(s,t\), violates \ref{onlyOne}, let
\begin{equation}
B := \left\{z\leq M \mid
  \delta(r,q,z) = 1\right\} \cap {\left\{d (s,t)\right\} \oplus w
  \naturals_0}
\end{equation} and add
\begin{equation}
  \sum_{z \in B} \delta(r, q, z)  \leq 1
\end{equation}
to the problem.
  
If \ref{onlyOne} holds, \(D\) implies a \(\ol{\Delta_S}\).  Calculate
\(\Delta_S\).  Fix a pair of vertices \(v,w\).  If \(\ol{\Delta_S
  (v,w)} = \Delta_S(v,w)\), do nothing.  Otherwise:
\begin{itemize}
  \item If \(\Delta_S < \infty\), then let \(P\) be the path between \(v\) and \(w\) in \(S\) and add the row
   \begin{equation}
    \sum_{a \in P} (m(a) - 1) \leq D \left(r,q, \sum_{a \in P} \frac{ d \left(a\right)}{w} - \frac{d(r,q)}{w}\right) - 1 \textrm{.}
    \end{equation} 
 \item If \(\Delta_S = \infty\), then let \(C \subseteq S\) be the cycle that includes \(v\) and add:
    \begin{equation}
      \sum_{a \in C \times (T \setminus C)} m(a) \geq \sum_{n \in L} D (r,q, n)
    \end{equation}
    to the problem.
\end{itemize}

\todo{Describe algorithm to find/produce a violated row.  Analyse its
  runtime.  Data structures?}

\todo{Row generation and elimination!  Or more specifically:
  Replacement by stricter constraints.  (Ignore until later.  Just add
  rows as needed.  Cplex will also take care of variables.)}

At the beginning there will be variables that are not present in any
constraint.  A solver like \textsc{Cplex} will only add them to the
problem as the constraints make them have an impact \ie as they show
up with at least one non-zero cöefficient in at least one constraint
or the objective function.

\subsection{Estimating the runtime of the oracle}

\todo{How fast can we find violated constraints?  I say, fast!  At
  least in polynomial time.  Perhaps even linear?}

\section{Balancing homogeneity with other objectives?}
Homogeneity is a very soft objective, \ie its value is hard to measure
in money.  Partly, the effort invested in achieving homogeneity is
intended to enhance adoption of the other optimizing work done at
department GSU1.

\section{Implementation}
Implementation will be done in the \textsc{Cplex}-framework existing at
department GSU1 of DB Mobility Logistics AG.
\section{Conclusion}

\end{document}
