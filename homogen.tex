\documentclass[a4paper]{amsart} % ams-article nehmen.
\usepackage{a4}
\usepackage{charter}
\usepackage[english]{babel}
\usepackage[utf8]{inputenc}
\usepackage{amssymb}
\usepackage{amsfonts}
\usepackage{units}
\usepackage{amsmath}

\usepackage{a4}

\author{Matthias Görgens}

% ---- LATIN ----
\def\etal{\emph{et~al.}}
\def\ie{\emph{i.e.}}
\def\eg{\emph{e.g.}}
\def\vitae{vit\ae{}}
\def\apriori{\emph{a~priori}}
\def\aposteriori{\emph{a~posteriori}}

% ---- FRENCH ----
\def\naive{na\"{\i}ve}
\def\Naive{Na\"{\i}ve}
\def\naively{na\"{\i}vely}	% Okay, I know, this isn't French.
\def\Naively{Na\"{\i}vely}


% Other
\def\Cpp{C\raisebox{0.5ex}{\tiny\bf++}}


\newcommand{\ol}[1]{\overline{#1}}
\newcommand{\todo}[1]{\marginpar{#1}}
%\newcommand{\todo}[1]{\footnote{#1}}

\DeclareMathOperator{\In}{in}
\DeclareMathOperator{\Out}{out}

%für ein/aus-gehende kanten
\newcommand{\ina}{\ensuremath{\vec{\In}}}
\newcommand{\outa}{\ensuremath{\vec{\Out}}}

%für knoten ein/aus-gehender kanten.
\newcommand{\inv}{\ensuremath{\dot{\In}}}
\newcommand{\outv}{\ensuremath{\dot{\Out}}}

\usepackage{hyperref}
\begin{document}

\title{Homogeneous vehicle schedules}

\begin{abstract}
  Although most railroad timetables remain in use for half a
  year, they tend to repeat themself each week.  Thus one can also
  repeat their implementations in rolling stock, the vehicle
  schedulings, each week.

  But timetables show even more structure: All the working days of the
  week are usually identical.  This work sets out to investigate how
  to preserve this similarity in vehicle schedulings.  We define
  homogeneous vehicle schedulings, establish a measure of partial
  homogenity and search for ways to efficiently find the most
  homogeneous vehicle scheduling for a given timetable.
\end{abstract}

\maketitle
\todo{unterschied zwischen fahrplan und umlaufplan im abstract ausarbeiten.}

\todo{Eigene konsistente Definitionen für Zug, Nutzfahrt etc:\\
Nutzfahrt = Train\\
Zug = set of trains, set}
\section{Some definitions}

Let \(G=(V,A)\) be a digraph of connected stations, \eg{} ordinary
railway stations, maintenance shops.  Now look at \label{trains}
\textit{trains} \((v, t_v, w, t_w) \in \left(V \times \mathbb{R}\right)^2 \)
where \(v\), \(t_v\) name the station and time of departure, and
\(w\), \(t_v\) name the station and time of arrival.  We call a set of
trains \(T \subseteq \left(V \times \mathbb{R}\right)^2\) a timetable.  (In
practise \(T\) can also be a multiset.)

We want to focus our attention to cyclic timetables.  A timetable \(T\)
is \label{cyclic} \textit{cyclic}, iff there exists a period \(w \in
\mathbb{R}^+\), so that for each train \((v, t_v, w, t_w) \in T\)
the trains \(\{(v, t_v+n w, w, t_w +n w) \mid n \in \mathbb{Z}\}\)
are also in \(T\).  We will identify cyclic timetables with their
image under the canonic morphism
\(\varphi \colon \left(V \times \mathbb{R}\right)^2
\to \left(V \times \mathbb{R}/{\left(w \mathbb{Z}\right)}\right)^2\).

A timetable only fixes the services offered by the railroad company.
But it does not specify where the locomotives and waggons in use come
from or where they go afterwards.  Naturally one wants to re-use
rolling stock from one train for another one.  So one seeks to
establish a digraph \(G_S=(T,S\subseteq T^2)\), where each train has
exactly one predecessor and exactly one successor.  Thus \(G_s\)
consists of cycles.

\todo{motivation.  warum keine einfacheren ansätze?}

In practise timetables repeat every week \ie{} \(w = 7 \textrm{ days}\).
This work will deal with timetables that show more structure: Fix an
interval \(d := \frac{w}{n}\) with \(n \in \mathbb{N}^+\) (\eg for
days in a week, let \(n=7\)).  We partition the trains into
equivalence classes that share the same arrival and departure
stations, and whose arrival and departure times only differ by an
integral multiply of \(w\) (\ie{} whole days):
\begin{equation}
  \left[\left( v, t_v, w, t_w \right) \right]_d := \{(v, t_v+n d, w, t_w +n d) \mid n \in \mathbb{Z}\}
\end{equation}
Let \(C\) be the set of all those classes.

Ideally each class has exactly one member for each day of the week.
That would allow \(w = 1 \textrm{day}\) instead of \(w = 7 \textrm{days}\)
and enable a schedule that repeats perfectly every day.  But more
typical me observe a lot of classes in a timetable with one member
each day from Monday to Friday and none on the weekends.  Or even less
regular arrangements.

We will define a distance function \(\Delta_S (s,t)\) on paths in
\(S\) as the sum of arc lengths \(d\) in the path between the
trains \(s\) and \(t\) or \(\infty\) if no such path exists.  We can
talk about \emph{the} path \(P \subseteq S\) between \(s\) and \(t\),
because \(G_S\) consists only of cycles.
\[\Delta_{S}\colon T \times T \to \mathbb{R}^+_0 \cup \{\infty\}\]
\begin{equation}
\label{defDelta}
\Delta_S (s,t) = \begin{cases}
\sum_{(v,w)\in P} d(v,w) & \textrm{if there is a path \( P \in S\) between \(s\) and \(t\)}\\
\infty & \textrm{else}
\end{cases}
\end{equation}

\todo{Bezeichnungen!  Pfeile über t?}

\todo{Mehrwöchige Fahrten!  Zurückgehen auf die ursprüngliche nicht-zyklische Version!}

Of course we have only moved the problem.  We still have to define the length of a single arc:  \(d(t, \ol{t})\) shall be the minimal time that a vehicle needs to provide \(t\) and be ready for the departure of \(\ol{t}\):
\[
d ((\cdot, t_{\textrm{dep}}, \cdot, t_{\textrm{arr}}), (\cdot, \ol{t_{\textrm{dep}}},\cdot, \cdot)  :=
\min (\{\ol{t_{\textrm{dep}}} - t_{\textrm{dep}} + n w \mid \ol{t_{\textrm{dep}}} - t_{\textrm{arr}} + nw \geq 0, n \in \mathbb{Z}\} \cap \mathbb{R}^+_0)
\]

In most cases this will be just the difference \(\ol{t_{\textrm{dep}}}
- t_{\textrm{dep}}\), but we may need to add a whole number of weeks
if \(\ol{t}\) departs before \{t\} arrives.  Also a more general approach
has to be taken when trains take more than one week from arrival to
departure.

To guarantee homogenity one might \naively{} propose that all members
of one class \(c_1\) only be succeeded by members of the same class
\(c_2\).  But that would not be practial, because \eg a class that is
offered all week would be banned from being connected to a class of trains that
are not offered on the weekend.

\todo{motivation.  faellt aus der luft.. zitat der begriffsbestimmung.}
So we introduce a weaker condition.  A schedule \(S\) is called
homogenous iff:
\begin{itemize}
\item Given classes \(z, \ol{z} \in C\)
\item and trains \(r_1, r_2 \in z\) and \(\ol{r}_1,\ol{r}_2 \in \ol{z}\)
\item with \( d (r_1, r_2) = d(\ol{r}_1, \ol{r}_2) \),
\item then either there is no path between neither \(r_1\) and
  \(\ol{r}_2\) nor \(r_2\) and \(\ol{r}_2\) in \(S\); or they are connected by paths of the same length:
\begin{equation}
\label{homoEq}
  \Delta_S (r_1, \ol{r_1}) = \Delta_S (r_2, \ol{r_2})\textrm{.}
\end{equation}
\end{itemize}


\todo{motivation fuer definition}
For arcs in a digraph \(G=(V,A)\) we will use the following notation:
\begin{align*}
\ina\colon  V &\to \mathcal{P}(A) \\
v &\mapsto \left(V \times \{v\}\right) \cap A\\
%\end{align*}
2%\begin{align*}
\outa\colon  V &\to \mathcal{P}(A) \\
v &\mapsto \left(\{v\} \times V\right) \cap A\\
%\end{align*}
%\begin{align*}
\inv\colon  V &\to \mathcal{P}(V) \\
v &\mapsto \left\{ u \colon \left(u,v\right) \in \ina(v) \right\}\\
%\end{align*}
%\begin{align*}
\outv\colon  V &\to \mathcal{P}(V) \\
v &\mapsto \left\{w \colon \left(v,w\right) \in \outa\left(v\right) \right\}\\
\end{align*}

\section{An infinite integer linear program}

Let \(T\) be a timetable.  And let \(G_\mathbb{S} := (T, \mathbb{S} =
T \times T) \) be its associated complete graph.  A schedule \(G_S = (T, S \subseteq \mathbb{S})\)
will be a subgraph of \(G_\mathbb{S}\).


For the integer program we introduce the variable \(m\) with the following domain and interpretation:
\todo{Notation: \(\cong\) means ``defined as'' or ``interprete as''}
\begin{align}
  m & \colon T \times T \to \{0,1\} \\
  m & \cong \mathbf{1}_{\mathbb{S}}
\end{align}

% subto matching_aus:
%       forall <v> in V:
%       	     sum <w> in V: umlauf[v,w] == 1;
% subto matching_ein:
%       forall <w> in V:
%       	     sum <v> in V: umlauf[v,w] == 1;

The following two equations capture the requirement that \(G_S\) consists of cycles.  For all \(r \in T\):
\begin{align}
  \label{matchBedingung}
  \sum_{a \in \ina(r)}  m(a) = \sum_{a \in \outa(r)} m(a) = 1
\end{align}

To reflect the values of \(\Delta_S\) in the integer program we use the binary variables \(\delta\):
\begin{align}
  \delta \colon T \times T \times \mathbb{R} & \to \{0,1\} \\
  \delta\left(r, q, \Delta_S\left(r,q\right)\right) &\cong 1 \\
  \label{onlyOne}
  \sum_{z \in \mathbb{R}} \delta(r, q, z) &\leq 1
\end{align}

For every path \(P \subseteq \mathbb{S}\) that starts in \(r\) and ends in \(q\) we have:
\begin{align}
\label{zwingHoch}
\sum_{a \in P} (m(a) - 1) \leq \delta \left(r,q, \sum_{a \in P} d \left(a\right)\right) - 1 \textrm{;}
\end{align}
\ie{} if \(P \subseteq S\) then length of \(P\) gives
the distance between its endpoints in \(S\).

And for every \(r\)-\(q\)-cut \(C \subseteq T\) (with \(r \in C\)):
\begin{align}
\label{zwingRunter}
  \sum_{a \in C \times (T \setminus C)} m(a) \geq \sum_{z \in R} \delta (r,q, z)
\end{align}

% \(G_S=(T,S\subseteq T^2)\)


% Aufbau: Modell oder Anwendung?
% Anwendung: Umlaufplanung etc, Abschätzung anderer Möglichkeiten.  Anwendung zur Motivation.
% Welcher Umfang zur Anwendung?  Literatur.  Allgemeine Beschreibung?  Irgendeine Referenz wird es schon geben.

% relation durch Übersetzung von Nutzfahrt ersetzen.

\todo{welche constraints?  labels und refs.  convex set vs gitterpunkte?}
Those constraints define a convex set.  Alas we can't use them
directly, since there are an infinite number of constraints.  But
fortunately they are constructed for use with row generation.  We will
show that there is a finite subset of constraints that define the same
polytope.  And we will show how to define a polynomial oracle to find
violated constraints.

\todo{Fuege die eigentliche homogenitaetsbedingung hinzu!}

\section{Finitizing the integer linear program}

Equation \ref{matchBedingung} holds for all \(r \in T\) and \(T\) is
already finite.  We declared \(\Delta_S\) to map into \(R^+_0 \cup
\{\infty\}\), but we can restrict the image even further:
\todo{notation: \(\oplus\) is the Minkowski sum.}
\begin{equation}
\label{einfacher}
\Delta_S (s,t) \in \{d (s,t)\} \oplus w \mathbb{N}_0
\end{equation}
\todo{Proof why \ref{einfacher} holds!}

Thus all \(\delta (s, t, z)\) with \(z \notin \{d (s,t)\} \oplus w
\mathbb{Z}\) vanish.  Furthermore because of the way \(\Delta_S\) is
defined in \ref{defDelta} we can bound it from above and below. Call
\[M := \sum_{(r,q) \in T \times T} d (r,q)\]
and note
\[0 \leq \Delta_S(s,t) \leq M \textrm{.}\]


Now for every \(s,t \in T\) the equations in an infinite number of
variables described by \ref{onlyOne} can be replaced by the finite:
\begin{align}
  \label{onlyOneFinite}
  \mathop{\sum_{z \in \left\{d (s,t)\right\} \oplus w \mathbb{N}_0}}_{z\leq M} \delta(r, q, z) & \leq 1 \\
  \label{zwingRunterFinite}
  \sum_{a \in C \times (T \setminus C)} m(a) & \geq \mathop{\sum_{z \in \left\{d (s,t)\right\} \oplus w \mathbb{N}_0}}_{z\leq M} \delta (r,q, z) \\
\end{align}

To simplify our notation we will introduce \(L := \{0, 1, \dots,
\left\lceil \frac{M}{w} \right\rceil \}\) and substitue \(D\) for
\(\delta\):
\begin{align}
  D \colon & T \times T \times L  \to \{0,1\} \\
          & (s,       t,        n)                              \mapsto \delta (s,t, d(s,t) + n w)
%  D (s,t,n) := 
\end{align}
Now we can rewrite \ref{onlyOneFinite} and \ref{zwingRunterFinite}:
\begin{align}
  \label{onlyOneFiniteN}
  \sum_{n \in L} D (r, q, n) & \leq 1 \\
  \label{zwingRunterFiniteN}
  \sum_{a \in C \times (T \setminus C)} m(a) & \geq \sum_{n \in L} D (r,q, n)
\end{align}

Putting \ref{zwingHoch} into the new form poses bigger
problem. \todo{hier erklaerung einfuegen.}  For every path \(P
\subseteq \mathbb{S}\) that starts in \(r\) and ends in \(q\) we have:
\begin{align}
%\sum_{a \in P} (m(a) - 1) \leq \delta \left(r,q, \sum_{a \in P} d \left(a\right)\right) - 1  \\
%\sum_{a \in P} (m(a) - 1) \leq D \left(r,q, \left[ \sum_{a \in P} d \left(a\right) \right] \right) - 1 \\
\sum_{a \in P} (m(a) - 1) \leq D \left(r,q, \frac{\left(\sum_{a \in P} d \left(a\right)\right) -d(r,q)}{w}\right) - 1  \\
\label{zwingHochFiniteN}
\sum_{a \in P} (m(a) - 1) \leq D \left(r,q, \sum_{a \in P} \frac{ d \left(a\right)}{w} - \frac{d(r,q)}{w}\right) - 1
\end{align}

\todo{Vorher koennten wir n fuer jeden Bogen unabhaeging waehlen.  Jetzt
  muessen wir eine Abhaengigkeit fordern.  Gluecklicherweise koennen wir das statisch
  bestimmen!

  Was ist die Laenge von \(P\) in terms of n in abhaengigkeit der
  zwischenschritte?  Und das hat auch geklappt in \ref{zwingHochFiniteN}}

\todo{Formuliere das verendlichte Problem konkret.  Und fuege die eigentliche homogenitaetsbedingung hinzu!}

\section{Row (and Column) Generation}
Even the finitized problem suffers from an exuberant size.
\todo{Abschaetzung der Groesse angeben.  O(Pfade + Cuts).}  But we can
start solving with only the equations \ref{matchBedingung} that
guarantee a flow free of sources and sinks and the variables \(m\).
All other variables are treated as parameters set to zero.

We analyse the solutions obtained.  The variables \(m\) specify a
graph made of cycles.  For every cycle we can get the distance
\(\Delta\) between all of its nodes by straight-forward counting in
time \(O(T^2)\) \todo{Details zum Zaehlen.} (or employing Bellman-Ford
etc).  All distances between nodes in different cycles are \(\infty\)
by definition \ref{defDelta}.  \todo{Perhaps use lazy evaluation to enumerate
  all flaws.  Pick the first.}

First check \ref{onlyOne} for every pair of vertices.  If a pair, say
\(s,t\), violates \ref{onlyOne}, let \[B := \left\{z\leq M \mid
  \delta(r,q,z) = 1\right\} \cap {\left\{d (s,t)\right\} \oplus w
  \mathbb{N}_0}\] and add
\begin{equation}
  \sum_{z \in B} \delta(r, q, z)  \leq 1
\end{equation}
to the problem.
  

If \ref{onlyOne} holds, \(D\) implies a \(\ol{\Delta_S}\).  Calculate
\(\Delta_S\).  Fix a pair of vertices \(v,w\).  If \(\ol{\Delta_S
  (v,w)} = \Delta_S(v,w)\), do nothing.  Otherwise:
\begin{itemize}
  \item If \(\Delta_S < \infty\), then let \(P\) be the path between \(v\) and \(w\) in \(S\) and add the row
   \begin{equation}
    \sum_{a \in P} (m(a) - 1) \leq D \left(r,q, \sum_{a \in P} \frac{ d \left(a\right)}{w} - \frac{d(r,q)}{w}\right) - 1 \textrm{.}
    \end{equation} 
 \item If \(\Delta_S = \infty\), then let \(C \subseteq S\) be the cycle that includes \(v\) and add:
    \begin{equation}
      \sum_{a \in C \times (T \setminus C)} m(a) \geq \sum_{n \in L} D (r,q, n)
    \end{equation}
    to the problem.
\end{itemize}

\todo{Describe algorithm to find/produce a violated row.  Analyse its runtime.  Data structures?}

\todo{Row generation and elimination!  Or more specifically: Replacement by stricter constraints.  (Ignore until later.  Just add rows as needed.  Cplex will also take care of variables.)}

At the beginning there will be variables that are not present in any
constraint.  The solver \textsc{Cplex} will only add them to the
problem as the constraints make them have an impact.

\end{document}
