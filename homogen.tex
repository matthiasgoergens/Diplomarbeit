\documentclass{amsart} % ams-article nehmen.
\usepackage{charter}
\usepackage[english]{babel}
\usepackage[utf8]{inputenc}
\usepackage{amssymb}
\usepackage{units}
\usepackage{amsmath}

% ---- LATIN ----
\def\etal{\emph{et~al.}}
\def\ie{\emph{i.e.}}
\def\eg{\emph{e.g.}}
\def\vitae{vit\ae{}}
\def\apriori{\emph{a~priori}}
\def\aposteriori{\emph{a~posteriori}}

% ---- FRENCH ----
\def\naive{na\"{\i}ve}
\def\Naive{Na\"{\i}ve}
\def\naively{na\"{\i}vely}	% Okay, I know, this isn't French.
\def\Naively{Na\"{\i}vely}


% Other
\def\Cpp{C\raisebox{0.5ex}{\tiny\bf++}}


\newcommand{\ol}[1]{\overline{#1}}

\begin{document}
\author{Matthias Görgens}
\title{Homogeneous vehicle schedules}

\abstract{

% unterschied zwischen fahrplan und umlaufplan.

  Although most railroad timetables remain in use for half a
  year, they tend to repeat themself each week.  Thus one can also
  repeat their implementations in rolling stock, the vehicle
  schedulings, each week.

  But timetables show even more structure: All the working days of the
  week are usually identical.  This work sets out to investigate how
  to preserve this similarity in vehicle schedulings.  We define
  homogeneous vehicle schedulings, establish a measure of partial
  homogenity and search for ways to efficiently find the most
  homogeneous vehicle scheduling for a given timetable.}

% vehicle scheduling

% Eigene konsistente Definitionen für Zug, Nutzfahrt etc.

% Nutzfahrt = Train
% Zug = set of trains, set
\maketitle
\section{Some definitions}

Let \(G=(V,A)\) be a digraph of connected stations, \eg{} ordinary
railway stations, maintenance shops.  Now look at \label{trains}
\textit{trains} \((v, t_v, w, t_w) \in \left(V,\mathbb{R}\right)^2 \)
where \(v\), \(t_v\) name the station and time of departure, and
\(w\), \(t_v\) name the station and time of arrival.  We call a set of
trains \(T \subseteq \left(V,\mathbb{R}\right)^2\) a timetable.  (In
practise \(T\) can also be a multiset.)

We want to focus our attention to cyclic timetables.  A timetable \(T\)
is \label{cyclic} \textit{cyclic}, iff there exists a period \(w \in
\mathbb{R}^+\), so that for each train \((v, t_v, w, t_w) \in T\)
the trains \(\{(v, t_v+n w, w, t_w +n w) \mid n \in \mathbb{Z}\}\)
are also in \(T\).  We will identify cyclic timetables with their
image under the canonic morphism
\(\varphi \colon \left(V,\mathbb{R}\right)^2
\to \left(V,\mathbb{R}/{\left(w \mathbb{Z}\right)}\right)^2\).


A timetable only fixes the services offered by the railroad company.
But it does not specify where the locomotives and waggons in use come
from or where they go afterwards.  Naturally one wants to re-use
rolling stock from one train for another one.  So one seeks to
establish a digraph \(G_S=(T,S\subseteq T^2)\), where each train has
exactly one predecessor and exactly one successor.  Thus \(G_s\)
consists of cycles.

% motivation.  warum keine einfacheren ansätze?

In practise timetables repeat every week \ie{} \(w = 7 \textrm{ days}\).
This work will deal with timetables that show more structure: Fix an
interval \(d := \frac{w}{n}\) with \(n \in \mathbb{N}^+\) (\eg for
days in a week, let \(n=7\)).  We partition the trains into
equivalence classes that share the same arrival and departure
stations, and whose arrival and departure times only differ by an
integral multiply of \(w\) (\ie{} whole days):
\begin{equation}
  \left[\left( v, t_v, w, t_w \right) \right]_d := \{(v, t_v+n d, w, t_w +n d) \mid n \in \mathbb{Z}\}
\end{equation}
Let \(C\) be the set of all those classes.

Ideally each class has exactly one member for each day of the week.
That would allow \(w = 1 \textrm{day}\) instead of \(w = 7 \textrm{days}\)
and enable a schedule that repeats perfectly every day.  But more
typical me observe a lot of classes in a timetable with one member
each day from Monday to Friday and none on the weekends.  Or even less
regular arrangements.

To guarantee homogenity one might \naively{} propose that all members
of one class \(c_1\) only be succeeded by members of the same class
\(c_2\).  But that would not be practial, because \eg a class that is
offered all week would be banned from being connected to a class of trains that
are not offered on the weekend.

% + definition of \Delta

% \Delta

So we introduce a weaker condition.  A schedule \(S\) is called
homogenous iff:
\begin{itemize}
\item Given classes \(z, \ol{z} \in C\)
\item and trains \(r_1, r_2 \in z\) and \(\ol{r}_1,\ol{r}_2 \in \ol{z}\),
% TODO: Gleicher Abstand?
\item then either there is no path between neither \(r_1\) and
  \(\ol{r}_2\) nor \(r_2\) and \(\ol{r}_2\) in \(S\); or they are connected by paths of the same length:
%todo: define \Delta first
\begin{equation}
\label{homoEq}
  \Delta_S (r_1, \ol{r_1}) = \Delta_S (r_2, \ol{r_2})
\end{equation}
\end{itemize}
with some distance function \(\Delta_S \colon \left(T,T\right) \to
\mathbb{R}^+_0 \mathbin{\dot{\cup}} \infty \) on paths in \(S\).

We will define \(\Delta_S (s,t)\) as the sum of arc lengths \(\delta\)
in the path between the trains \(s\) and \(t\) or \(\infty\) if no
such path exists.  We can talk about \emph{the} path \(P \subseteq S\)
between \(s\) and \(t\), because \(G_S\) consists only of cycles.
\[\Delta_S (s,t) = \sum_{(v,w)\in P} \delta(v,w)\]

% Bezeichnungen!  Pfeile über t?

% Mehrwöchige Fahrten!

Of course we have only moved the problem.  We still have to define the length of a single arc:  \(\delta(t, \ol{t})\) shall be the minimal time that a vehicle needs to provide \(t\) and be ready for the departure of \(\ol{t}\):
\[
\delta ((\cdot, t_{\textrm{dep}}, \cdot, t_{\textrm{arr}}), (\cdot, \ol{t_{\textrm{dep}}},\cdot, \cdot)  :=
\min (\{\ol{t_{\textrm{dep}}} - t_{\textrm{dep}} + n w \mid \ol{t_{\textrm{dep}}} - t_{\textrm{arr}} + nw \geq 0, n \in \mathbb{Z}\} \cap \mathbb{R}^+_0)
\]

In most cases this will be just the difference \(\ol{t_{\textrm{dep}}} - t_{\textrm{dep}}\), but we may need to add a whole number of weeks if \(\ol{t}\) departs before \{t\} arrives.






% Aufbau: Modell oder Anwendung?
% Anwendung: Umlaufplanung etc, Abschätzung anderer Möglichkeiten.  Anwendung zur Motivation.
% Welcher Umfang zur Anwendung?  Literatur.  Allgemeine Beschreibung?  Irgendeine Referenz wird es schon geben.

% relation durch Übersetzung von Nutzfahrt ersetzen.


% Let \(T = \mathbb{R}/{n \mathbb{Z}} \)
% be the time that 


\end{document}

% DSL: states; uebergaenge als zertifikate, wenn nötig.
% was ist mit zufälligen einflüssen? -> zufall spielt mit.

% data structure:
% [region: {[pop: {mv}], lscape, adv, nachbarn}

% Runde DSL? top-down?
% pop-growth
% pop-move

% event

% advances
% upkeep (cities, tribes)