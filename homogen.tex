\documentclass{article} % ams-article nehmen.
\usepackage[english]{babel}
\usepackage[utf8]{inputenc}
\usepackage{amssymb}
\usepackage{units}

\begin{document}
\author{Matthias Görgens}
\title{Homogeneous train schedules}
\maketitle
\abstract{Although most railroad schedules remain in use for half a year,
  they tend to repeat themself each week.  Thus one can also repeat
  their implementations, the vehicle schedulings, each week.

  But railroad schedules show even more structure: All the working
  days of the week are usually identical.  This work sets out to
  investigate how to transport this similarity in relations to their
  vehicle scheduligs.  We define homogeneous vehicle schedulings,
  establish a measure of partial homogenity and search for ways to
  efficiently find the most homogeneous vehicle scheduling for a given
  schedule.}

% Umlaufplan? flow plan.  Aber es gibt sicher bessere Worte bzw sogar ein offizielles.
% vehicle scheduling, (rotation planning,) rolling stock (niederländer, aber auch noch Waggons).

% Eigene konsistente Definitionen für Zug, Nutzfahrt etc.
\section{Some definitions}

Let \(G=(V,A)\) be a digraph of connected stations, i.e. railway
stations, maintenance shops.  Now look at \label{relation}
\textit{relations} \((v, t_v, w, t_w) \in \left(V,\mathbb{R}\right)^2
\) where \(v\), \(t_v\) name the station and time of departure, and
\(w\), \(t_v\) name the station and time of arrival.  We call a set of
relations \(S \subseteq \left(V,\mathbb{R}\right)^2\) a schedule.

We want to focus our attention to cyclic schedules.  A schedule \(S\)
is \label{cyclic} \textit{cyclic}, iff there exists a period \(w \in
\mathbb{R}^+\), so that for each relation \((v, t_v, w, t_w) \in S\)
the relations \(\{(v, t_v+n w, w, t_w +n w) \mid n \in \mathbb{Z}\}\)
are also in \(S\).  For a cyclic schedule \(S\) we will identify \(S\)
with its image under the canonic morphism \(\varphi \colon
\left(V,\mathbb{R}\right)^2 \to \left(V,\mathbb{R}/{w
    \mathbb{Z}}\right)^2\).

In practise schedules repeat every week i.e. \(w = 7 \unit{days}\).

This work will deal with schedules that show more structure: Fix an
interval \(d := \frac{w}{n}\) with \(n \in \mathbb{N}^+\); in addition
to the schedule \(S\) introduce the set of trains \(T \subseteq
\mathcal{P}(S)\) partitioning the relations of the schedule.
Relations in each \textit{train} \(z \in T\) all share the same arrival
and departure stations, also their arrival and departure times only
differ by an integral multiply of \(w\).  That is
\begin{equation}
  z \subseteq [(v, t_v, w, t_w )]_d := \{(v, t_v+n d, w, t_w +n d) \mid n \in \mathbb{Z}\}
\end{equation}



% Aufbau: Modell oder Anwendung?
% Anwendung: Umlaufplanung etc, Abschätzung anderer Möglichkeiten.  Anwendung zur Motivation.
% Welcher Umfang zur Anwendung?  Literatur.  Allgemeine Beschreibung?  Irgendeine Referenz wird es schon geben.




% Let \(T = \mathbb{R}/{n \mathbb{Z}} \)
% be the time that 



\end{document}